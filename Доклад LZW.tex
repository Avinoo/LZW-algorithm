\documentclass{article}
\usepackage[russian]{babel}
\usepackage[letterpaper,top=2cm,bottom=2cm,left=3cm,right=3cm,marginparwidth=1.75cm]{geometry}
\usepackage[utf8]{inputenc}
\usepackage{amsmath}
\usepackage{graphicx}
\usepackage{float}
\usepackage[colorlinks=true, allcolors=blue]{hyperref}

\title{Алгоритм Лемпеля-Зива-Велча}
\author{Анна Виноходова}
\date{2022 г.}

\begin{document}
\maketitle

\section{Введение}

\subsection{Суть и назначение}

Алгоритм Лемпеля-Зива-Велча – это универсальный алгоритм сжатия данных без потерь.

\subsection{Авторство}

Алгоритм был опубликован Терри Велчем в 1984 году в качестве улучшенного алгоритма LZ78, опубликованного Абрахамом Лемпелем и Якобом Зивом в 1978 г., и получил название LZW.

\subsection{История развития}

Наиболее распространенными модификациями алгоритма LZW являются: LZC (Lempel-Ziv Compress, 1985 г.), LZT (Lempel-Ziv-Tischer, 1985 г.), LZMW (Lempel-Ziv-Miller-Wegman, 1985 г.) и LZAP (Lempel-Ziv All Prefixes, 1988 г.).

\subsection{Состояние, реализация}

На момент своего появления алгоритм LZW стал первым широко используемым на компьютерах методом сжатия данных.

\subsection{Перспектива использования}

В настоящее время используется в файлах формата TIFF, PDF, GIF, PostScript и других, а также отчасти во многих популярных программах сжатия данных (ZIP, ARJ, LHA).

\section{Описание метода}

\subsection{Формальное}

При кодировании сообщения стандартный алгоритм LZW создает словарь строк. Каждой строке присваивается уникальных 12-битный код. Сначала словарь заполняется всеми односимвольными строками, содержащимися в сообщении. Максимальный размер словаря составляет 4096 строк с кодами. Алгоритм считывает текст сообщения посимвольно слева направо и ищет максимальную строку, которой нет в словаре – WK, где W – строка, имеющаяся в словаре, а K – символ, следующий за ней в сообщении. Найденная строка WK вносится в словарь и ей присваивается уникальный код, программа выводит код строки W, а следующая рассматриваемая строка начинается символа K. В случае переполнения словаря он продолжает использоваться без добавления строк.
При декодировании сообщения алгоритм создает словарь фраз идентичный тому, что создавался при кодировании. На вход необходимо только закодированное сообщение. Процесс декодирования имитирует процесс кодирования и может происходить одновременно с ним.

\subsection{Математическое}

Кодирование:
\begin{enumerate}
\item Инициализировать словарь со всеми односимвольными строками входного сообщения;
\item Инициализировать строку W и присвоить ей первый символ входного сообщения;
\item Если КОНЕЦ СООБЩЕНИЯ, то вывести код для W и завершить алгоритм.
Считать очередной символ K из входного сообщения;
\item Если фраза WK уже есть в словаре, то присвоить входной фразе W значение WK и перейти к шагу 3;
\item Иначе выдать код W, добавить WK в словарь, присвоить входной фразе W значение K и перейти к шагу 3;
\end{enumerate}

\subsection{Пример}

Входное сообщение, которое необходимо закодировать, имеет вид:
ababcbababaaaaaaa
Программа инициализирует словарь, содержащий 3 односимвольных строки (a; b; c) и присваивает им уникальные 12-битные коды (1;2;3). Каждая новая уникальная фраза заносится в словарь (добавим фразу ab). Ей присваивается уникальный код (ab присвоим код 4). На выход поступает код фразы, которая короче на один символ и уже присутствует в словаре (выведем код фразы a - 1).

\begin{table}[h]
\centering
\begin{tabular}{ |c|c|c| } 
 \hline
 Новая фраза & Десятичный код & Вывод \\
 \hline\hline
 a & 1 &  \\
 \hline
 b & 2 &  \\
 \hline
 c & 3 &  \\
 \hline
 ab & 4 & 1 \\
 \hline
 ba & 5 & 2 \\
 \hline
 abc & 6 & 4 \\
 \hline
 cb & 7 & 3 \\
 \hline
 bab & 8 & 5 \\
 \hline
 baba & 9 & 8 \\
 \hline
 aa & 10 & 1 \\
 \hline
 aaa & 11 & 10 \\
 \hline
 aaaa & 12 & 11 \\
 \hline
 - & - & 1 \\
 \hline
\end{tabular}
\caption{\label{tab:widgets}Пример процесса кодирования}
\end{table}

Таким образом закодированное сообщение получится следующим:
124358110111

\section{Формальная постановка задачи}

Написать программу, осуществляющую кодирование входного сообщения по алгоритму Лемпеля-Зива-Велча. Программа должна инициализировать динамический словарь всех односимвольных фраз входного сообщения и в процессе кодировки пополнять его новыми фразами. Каждой фразе должен присваиваться уникальный 12-битный код. Считывание текста сообщения происходит посимвольно слева направо. 

Формат входного файла:
Входной файл содержит текст в кодировке UTF-8 без разделения на абзацы.

Формат выходного файла:
Выходной файл содержит строку целых чисел без разделителей.

\section{Список литературы}


\end{document}
