\documentclass[a4paper]{article}
\usepackage[russian]{babel}
\usepackage[utf8]{inputenc}
\usepackage{graphicx}
\usepackage{float}
\usepackage[colorlinks=true, allcolors=blue]{hyperref}
\usepackage{verbatim}
\usepackage{alltt}
\usepackage[14pt]{extsizes}
\usepackage{setspace,amsmath}
\usepackage[left=20mm, top=15mm, right=15mm, bottom=15mm, nohead, footskip=10mm]{geometry} 
\begin{document}
\begin{center}
\hfill \break
\large{МИНИСТЕРСТВО НАУКИ И ВЫСШЕГО ОБРАЗОВАНИЯ РОССИЙСКОЙ ФЕДЕРАЦИИ}\\
\footnotesize{Федеральное государственное автономное образовательное учреждение высшего образования}\\ 
\footnotesize{«Дальневосточный федеральный университет»}\\
\small{\textbf{(ДВФУ)}}\\
\hfill \break
\normalsize{ИНСТИТУТ МАТЕМАТИКИ И КОМПЬЮТЕРНЫХ ТЕХНОЛОГИЙ}\\
 \hfill \break
\normalsize{Департамент математического и компьютерного моделирования}\\
\hfill\break
\hfill \break
\hfill \break
\hfill \break
\large{Адаптивный алгоритм Лемпеля-Зива-Велча}\\
\hfill \break
\hfill \break
\hfill \break
\normalsize{Доклад\\
\hfill \break
Направление подготовки 09.03.03 Прикладная информатика\\
\hfill \break
Профиль «Прикладная информатика в компьютерном дизайне»}\\
\hfill \break
\hfill \break
\end{center}
\normalsize{} \hfill \break
\hfill \break
\normalsize{ 
\begin{tabular}{cccc}
Обучающийся & \underline{\hspace{3cm}} & &А.А. Виноходова \\\\
Руководитель & \underline{\hspace{3cm}}& доцент ИМКТ &А.С. Кленин \\\\
\end{tabular}
}\\
\hfill \break
\hfill \break
\begin{center} Владивосток 2022 \end{center}
\thispagestyle{empty} 
\newpage
     
    \tableofcontents 
\newpage
\newpage
\section{Введение}

\subsection{Суть и назначение}

Алгоритм LZW (алгоритм Лемпеля-Зива-Велча) – это универсальный алгоритм сжатия данных без потерь. Он предназначен для кодирования текста и графических изображений. По своей сути алгоритм напоминает канал связи тем, что переводит входную последовательность байт в более удобный для хранения и передачи формат, а затем декодирует ее обратно в точную копию входного сообщения.

\subsection{Авторство}

Алгоритм LZW был опубликован Терри А. Велчем в 1984 году в качестве улучшения алгоритма LZ78, описанного ранее Абрахамом Лемпелем и Якобом Зивом. Название LZW является аббревиатурой фамилий создателей (англ. Lempel-Ziv-Welch).


\subsection{История развития}

Алгоритм Лемпеля-Зива-Велча входит в семейство алгоритмов словарного сжатия данных LZ, берущее начало от алгоритмов LZ77 и LZ78. Наиболее распространенными модификациями алгоритма LZW являются: LZC (Lempel-Ziv Compress, 1985 г.), LZT (Lempel-Ziv-Tischer, 1985 г.), LZMW (Lempel-Ziv-Miller-Wegman, 1985 г.) и LZAP (Lempel-Ziv All Prefixes, 1988 г.).

\subsection{Состояние, реализация}

На момент своего появления алгоритм LZW стал первым широко используемым на компьютерах методом сжатия данных. Он стал популярен благодаря своим невысоким требованиям к програмному обеспечению и сравнительно простой реализации.

\subsection{Перспектива использования}

В настоящее время используется в файлах таких форматов, как TIFF, PDF, GIF, PostScript, а также во многих известных архиваторах, в том числе ZIP, ARJ, LHA.

\newpage
\section{Метод}

\subsection{Формальное описание метода}
\subsubsection{Словарь}
Инициализируется динамический словарь. Словарь включает все символы используемого алфавита (в данном случае алфавит представляет из себя первые 256 символов из таблицы ascii). Коды в словаре изначально имеют длину 8 бит (т.е. словарь включает 256 записей). 
\subsubsection{Кодирование}
Алгоритм считывает текст сообщения посимвольно слева направо и ищет максимальную строку, которой нет в словаре – WK, где W – строка, имеющаяся в словаре, а K – символ, следующий за ней в сообщении. Найденная строка WK вносится в словарь и ей присваивается уникальный код, программа выводит код строки W, а следующая рассматриваемая строка начинается символа K. На выход при этом поступает код строки из словаря, которая короче найденной на 1 символ.
\subsubsection{Заполнение словаря}
По мере добавления записей в словарь в случае переполнения длина новых кодов увеличивается на 1 бит (т.е. как только потребуется больше 256 записей в словаре, длина новых кодов становится 9 бит, а размер словаря увеличивается до 512 записей). Максимальная длина кодов составляет 16 бит (65536 записей в словаре). 
\subsubsection{Переполнение словаря}
По достижении максимального количества записей словарь перестает пополняться и используется дальше без изменений.
\subsubsection{Декодирование}
При декодировании сообщения алгоритм создает словарь фраз идентичный тому, что создавался при кодировании. На вход требуется только закодированное сообщение. Процесс декодирования имитирует процесс кодирования и может происходить одновременно с ним.

\subsection{Псевдокод}
\subsubsection{Кодирование:}
\begin{enumerate}
\item Инициализировать начальный словарь, содержащий все возможные символы;
\item Инициализировать строку W и присвоить ей первый символ входного сообщения;
\item Если КОНЕЦ СООБЩЕНИЯ, то вывести код для W и завершить алгоритм.
Считать очередной символ K из входного сообщения;
\item Если фраза WK уже есть в словаре, то присвоить входной фразе W значение WK и перейти к шагу 2;
\item Иначе выдать код W, добавить WK в словарь, присвоить входной фразе W значение K и перейти к шагу 2.
\end{enumerate}

\subsubsection{Декодирование:}
\begin{enumerate}
\item Инициализировать начальный словарь, содержащий все возможные символы;
\item Инициализировать строку W и присвоить ей первый символ декодируемого сообщения;
\item Считать очередной код K из сообщения;
\item Если КОНЕЦ СООБЩЕНИЯ, то вывести символ, соответствующий коду W, иначе:
\item Если фразы под кодом WK нет в словаре, вывести фразу, соответствующую коду W, а фразу с кодом WK занести в словарь;
\item Иначе присвоить входной фразе код WK и перейти к Шагу 2.
\end{enumerate}

\newpage
\subsection{Пример}
\subsubsection{Кодирование}
Входное сообщение, которое необходимо закодировать, имеет вид:
ababcbababaaaaaaa
Инициализируется словарь односимвольных строк, содержащий 256 записей. В Таблице 1 приведен фрагмент исходного словаря, который будет использоваться в примере. 
\begin{table}[h]
\centering
\begin{tabular}{ |c|c| } 
\hline
 Символ & Код\\
 \hline\hline
 ... & ... \\
\hline
 a & 97  \\
 \hline
 b & 98  \\
 \hline
 с & 99  \\
\hline
 ... & ... \\
\hline
 & 255 \\
\hline
 ... & ... \\
\hline
\end{tabular}
\caption{\label{tab:widgets}Фрагмент исходного словаря при кодировании}
\end{table}

Каждая новая фраза (в данном случае начиная с ab) заносится в словарь и ей присваивается уникальный код (ab присваивается 256). На выход поступает код записи, которая на символ короче данной (код фразы a - 97)

\begin{table}[h]
\centering
\begin{tabular}{ |c|c|c| } 
 \hline
 Новая фраза & Код & Вывод \\
 \hline\hline
 a & 97 &  \\
 \hline
 b & 98 &  \\
 \hline
 c & 99 &  \\
 \hline
 ab & 256 & 97 \\
 \hline
 ba & 257 & 98 \\
 \hline
 abc & 258 & 256 \\
 \hline
 cb & 259 & 99 \\
 \hline
 bab & 260 & 257 \\
 \hline
 baba & 261 & 260 \\
 \hline
 aa & 262 & 97 \\
 \hline
 aaa & 263 & 262 \\
 \hline
 aaaa & 264 & 263 \\
 \hline
 - & - & 97 \\
 \hline
\end{tabular}
\caption{\label{tab:widgets}Процесс кодирования}
\end{table}

Таким образом закодированное сообщение получится следующим:
97 98 256 99 257 260 97 262 263 97. Для наглядности в качестве разделителей выбраны пробелы.

\newpage
\section{Формальная постановка задачи}
\begin{enumerate}
\item Изучить алгоритм LZW на основе литературных источников, описать его в форме научного доклада. 

\item Реализовать адаптивную версию алгоритма LZW, позволяющую по мере заполнения словаря увеличивать длину кодов от 8 до 16 бит. 

Формат входного файла:
Текстовый файл формата .txt в количестве символов N, включая строчные и прописные латинские символы, символы кирилицы, пробелы, переносы строки, знаки пунктуации. 

Формат выходных файлов:
Два текстовых файла формата .txt. Первый содержит закодированный текст входного файла, каждый код в файле располагается на отдельной строчке. Второй содержит декодированный текст. Входной файл и файл с декодированным текстом должны совпадать.

Ограничения:
($1<=N<=2^2^0$).

\item Исследовать алгоритм на предмет зависимости степени сжатия файла от его размера.

\itemРезультаты работы выложить в репозиторий GitHub
\end{enumerate}
\newpage

\section{Тестирование}
\subsection{Корректность}
Для проверки корректности работы алгоритма представлены тесты следующего формата: в дирекетории Tests находятся 3 директории (in, encoded, decoded). В директории in находится последовательность из n тестов. Нумерация тестов начинается с 0 и не имеет пропусков. Название теста содержит только его порядковый номер. В директории encoded в результате работы функции encode создаются текстовые файлы, названия которых соответствующих порядковому номеру теста. В директории decoded в результате работы функции decode также создаются текстовые файлы, названия которых соответствующих порядковому номеру теста. Проверка на корректность работы программы в конкретном тесте считается пройденной, если если соответственные файлы в директории in и директрории decoded равны.

\newpage
\section{Исследование}
\subsection{Коэффициент сжатия}
Для оценки степени сжатия после выполнения теста высчитывается коэффициент сжатия по формуле $k = S_0 / S_c$, где $S_0$ - размер исходного файла, $S_c$ - размер сжатого файла. Чем больше коэффициент сжатия, тем эффективнее сжатие. Если $k$ будет равен единице, то сжатый файл по размеру равен исходному. Если $k$ будет меньше нуля, то сжатый файл по размеру больше исходного.
\subsection{Зависмость степени сжатия от размера файла}
Исследование зависимости степени сжатия от размера файла производится на тестах, содержание которых сгенерировано случайно. Тесты содержат $2^5$, $2^1^0$, $2^1^5$ и $2^2^0$ латинских строчных символов таблицы ascii.

\newpage
\section{Список литературы}

\begin{enumerate}
\item Welch T. A. A technique for high-performance data compression // Computer. — 1984. — Т. 6, № 17. — С. 8–19. — doi:10.1109/MC.1984.1659158.
\item Dinsky [Электронный ресурс] https://youtu.be/XsllPSupzy4
\item Arnold R., Bell T.[en]. A corpus for the evaluation of lossless compression algorithms // IEEE Data Compression Conference. — 1997. — С. 201–210. — doi:10.1109/DCC.1997.582019.
\item Bell T.[en], Witten I. H.[en], Cleary J. G.[en]. Modeling for text compression // ACM Computing Surveys[en]. — 1989. — Т. 21, № 4. — С. 557–591. — doi:10.1145/76894.76896.
\item Charikar M., Lehman E., Lehman A., Liu D., Panigrahy R., Prabhakaran M., Sahai A., shelat a. The smallest grammar problem // IEEE Transactions on Information Theory[en]. — 2005. — Т. 51, № 7. — С. 2554—2576. — doi:10.1109/TIT.2005.850116.
\item De Agostino S., Silvestri R. A worst-case analysis of the LZ2 compression algorithm // Information and Computation[en]. — 1997. — Т. 139, № 2. — С. 258–268. — doi:10.1006/inco.1997.2668.
\item De Agostino S., Storer J. A. On-line versus off-line computation in dynamic text compression // Information Processing Letters[en]. — 1996. — Т. 59, № 3. — С. 169–174. — doi:10.1016/0020-0190(96)00068-3.
\item Hucke D., Lohrey M., Reh C. P. The smallest grammar problem revisited // String Processing and Information Retrieval (SPIRE). — 2016. — Т. 9954. — С. 35–49. — doi:10.1007/978-3-319-46049-9-4.
\item Lempel A., Ziv J. Compression of individual sequences via variable-rate coding // IEEE Transactions on Information Theory[en]. — 1978. — Т. 24, № 5. — С. 530–536. — doi:10.1109/TIT.1978.1055934.
\item Miller V. S[en], Wegman M. N.[en]. Variations on a theme by Ziv and Lempel // Combinatorial algorithms on words. — 1985. — Т. 12. — С. 131–140. — doi:10.1007/978-3-642-82456-2-9.
\item Sheinwald D. On the Ziv-Lempel proof and related topics // Proceedings of the IEEE[en]. — 1994. — Т. 82, № 6. — С. 866–871. — doi:10.1109/5.286190.
\item Storer J. A. Data Compression: Methods and Theory. — New York, USA: Computer Science Press, 1988. — 413 с. — ISBN 0-7167-8156-5.
\item Ziv J. A constrained-dictionary version of LZ78 asymptotically achieves the finite-state compressibility with a distortion measure // IEEE Information Theory Workshop. — 2015. — С. 1–4. — doi:10.1109/ITW.2015.7133077.
\item Adobe Systems Incorporated. Document management — Portable document format — Part 1: PDF 1.7 (англ.). PDF 1.7 specification. Adobe (1 июля 2008). Дата
\item Wikipedia — Lempel — Ziv — Welch
\item Семенюк В.В. — Экономное кодирование дискретной информации
\item Метод LZW — сжатия данных — алгоритмы и методы
\item Алгоритмы сжатия и компрессии
\item Алгоритм LZW — Понятие алгоритма
\item Вирт Н. Алгоритмы и структуры данных/Н. Вирт. М.: Мир,1989.
\item Сибуя М. Алгоритмы обработки данных/М. Сибуя, Т. Ямамото. М.: Мир,1986.
\item Костин А.Е. Организация и обработка структур данных в вычислительных системах: учеб.пособ. для вузов/А.Е. Костин, В.Ф. Шаньгин . М.: Высш.шк., 1987.
\item Кнут Д. Искусство программирования для ЭВМ.Т.1: Основные алгоритмы:пер. с англ./Д. Кнут. М.:Мир,1978.
\item Кнут Д. Искусство программирования для ЭВМ.Т.3: Сортировка и поиск.: пер. с англ./Д.Кнут. М.:Мир,1978.
\item Кормен Т. Алгоритмы: построение и анализ./Т. Кормен, Ч. Лейзерсон, Р.Ривест. М.: МЦНМО, 2000
\item Кричевский Р.Е. Сжатие и поиск информации/Р.Е. Кричевский. М.: Радио и связь, 1989
\item Интернет ресурс. https://habr.com/ru/post/132683/
\end{enumerate}

\end{document}
